\documentclass[a4paper,11pt]{article}

\usepackage[utf8]{inputenc}
\usepackage[T1]{fontenc}
\usepackage[margin=2.5cm]{geometry}
\usepackage{parskip}
\usepackage{titlesec}
\usepackage{enumitem}
\usepackage{xcolor}
\usepackage{booktabs}
\usepackage{hyperref}

\usepackage{tikz}
\usepackage{pagecolor}
\usepackage{array}
\usepackage{tabularx}
\usepackage{amssymb}

% ── Colors ───────────────────────────────────────────────
\definecolor{bg}{HTML}{07070f}
\definecolor{ink}{HTML}{e2e2f0}
\definecolor{mid}{HTML}{a0a0c0}
\definecolor{soft}{HTML}{555577}
\definecolor{rule}{HTML}{1e1e35}
\definecolor{purple}{HTML}{a78bfa}
\definecolor{purpledark}{HTML}{6366f1}
\definecolor{green}{HTML}{4ade80}
\definecolor{yellow}{HTML}{facc15}
\definecolor{rowA}{HTML}{0d0d1a}
\definecolor{rowB}{HTML}{111120}

% ── Page background ──────────────────────────────────────
\pagecolor{bg}
\color{ink}

% ── Hyperref ─────────────────────────────────────────────
\hypersetup{
  colorlinks=true,
  linkcolor=purple,
  urlcolor=purple,
  pdftitle={Projeto GuroZord},
  pdfauthor={GuroZord},
}

% ── Section formatting ───────────────────────────────────
\titleformat{\section}
  {\large\bfseries\color{ink}}
  {}{0em}{}[\color{rule}\titlerule]

\titleformat{\subsection}
  {\normalsize\bfseries\color{purple}}
  {}{0em}{}

\titlespacing{\section}{0pt}{20pt}{8pt}
\titlespacing{\subsection}{0pt}{14pt}{4pt}

\pagestyle{empty}

% ── TikZ full logo ───────────────────────────────────────
\newcommand{\gurozordlogo}[1][1]{%
  \begin{tikzpicture}[scale=#1, baseline=0]
    % background
    \fill[black!92, rounded corners=6pt] (0,0) rectangle (5.4,1.2);
    \draw[purple, opacity=0.12, rounded corners=6pt, line width=0.6pt]
      (0.04,0.04) rectangle (5.36,1.16);

    % eye icon
    \draw[purple, opacity=0.3, rounded corners=2.5pt, line width=0.6pt]
      (0.15,0.1) rectangle (1.05,1.0);
    \draw[purple, line width=0.9pt]
      (0.22,0.55) .. controls (0.6,0.18) and (0.6,0.92) .. (0.98,0.55)
                  .. controls (0.6,0.92) and (0.6,0.18) .. (0.22,0.55);
    \draw[purple, line width=0.7pt] (0.6,0.55) circle (0.14);
    \fill[purple] (0.6,0.55) circle (0.055);

    % wordmark
    \node[anchor=base west, inner sep=0pt] at (1.2, 0.38) {%
      \fontsize{13}{13}\selectfont\bfseries\color{purple}GuroZord%
    };

    % divider
    \draw[purple, opacity=0.25, line width=0.4pt] (1.2,0.26) -- (5.25,0.26);

    % tagline
    \node[anchor=base west, inner sep=0pt] at (1.22, 0.14) {%
      \fontsize{4.5}{4.5}\selectfont\color{purple!50}%
      \MakeUppercase{From chaos came order}%
    };
  \end{tikzpicture}%
}

\begin{document}

% ── Title block ──────────────────────────────────────────
\noindent
{\fontsize{28}{34}\selectfont\bfseries\color{ink} Projeto GuroZord}\\[4pt]
{\small\color{soft}\MakeUppercase{Requisitos Funcionais · v1.0}}

\vspace{12pt}
\noindent\gurozordlogo[1.5]

\vspace{10pt}
{\color{rule}\hrule}
\vspace{20pt}

% ── Contexto ─────────────────────────────────────────────
\section{Contexto}

Bot simples de moderação de grupos de WhatsApp.

{\color{mid}\itshape
Origem da ideia: fechar o grupo em horário configurável, reabrir pela manhã,
exibir ranking de interações e incentivar a participação de todos os membros.
}

% ── Requisitos Funcionais ─────────────────────────────────
\section{Requisitos Funcionais}

\subsection{RF1 · Grupo Fechado com Horário Configurável}
{\color{mid}\itshape
Fechar o grupo para mensagens de não-admins em horário configurável (ex.: 01h)
e reabrir automaticamente em outro horário (ex.: 10h). Apenas administradores
podem enviar mensagens fora do horário ativo.
}

\subsection{RF2 · Leaderboard de Engajamento}
{\color{mid}\itshape
Exibir ranking com o Top 10 de membros com mais mensagens enviadas no período.
Pode ser acionado via comando \texttt{/top} ou enviado automaticamente em horário
fixo na reabertura do grupo.
}

\subsection{RF3 · Banimento Automático por Inatividade}
{\color{mid}\itshape
Monitorar a atividade dos membros e aplicar política de inatividade progressiva:
}

\begin{itemize}[leftmargin=1.5em, itemsep=2pt]
  \item \textbf{\color{ink}RF3.1} {\color{mid}— Aviso via mensagem privada após 3 dias de inatividade.}
  \item \textbf{\color{ink}RF3.2} {\color{mid}— Banimento automático do grupo após 7 dias de inatividade.}
\end{itemize}

\subsection{RF4 · Sistema de Avisos (Warnings)}
{\color{mid}\itshape
Sistema de advertências para usuários que violarem regras. Cada aviso é registrado
e pode acumular para resultar em punições automáticas (mute, kick, ban).
}

\subsection{RF5 · Filtro de Palavras e Conteúdo Sensível}
{\color{mid}\itshape
Moderação automática de conteúdo indesejado no grupo:
}

\begin{itemize}[leftmargin=1.5em, itemsep=2pt]
  \item \textbf{\color{ink}RF5.1} {\color{mid}— Bloquear automaticamente palavrões excessivos.}
  \item \textbf{\color{ink}RF5.2} {\color{mid}— Detectar e remover links suspeitos.}
  \item \textbf{\color{ink}RF5.3} {\color{mid}— Bloquear divulgação de outros grupos.}
  \item \textbf{\color{ink}RF5.4} {\color{mid}— Detectar e punir ataques pessoais entre membros.}
\end{itemize}

% ── Plano de Entrega ─────────────────────────────────────
\section{Plano de Entrega}

{\color{mid}\itshape
Desenvolvedor solo · Prazo final: 28/02/2026
}

\vspace{10pt}

% ── Fase 1 ───────────────────────────────────────────────
\subsection{Fase 1 · MVP — Controle de Horário \hfill {\normalfont\small\color{soft}20–21/02}}

{\color{mid}
Implementar a lógica central de abertura e fechamento do grupo.
Esta fase entrega valor imediato e valida a integração com a API do WhatsApp.
}

\begin{itemize}[leftmargin=1.5em, itemsep=3pt]
  \item {\color{ink}RF1} — Agendador de abertura/fechamento com horários configuráveis
  \item {\color{ink}RF1} — Restrição de mensagens para não-admins fora do horário ativo
  \item {\color{ink}—} — Configuração via arquivo \texttt{.env} ou comando \texttt{/config}
\end{itemize}

\vspace{4pt}
{\small\color{green}$\checkmark$ \textbf{Marco:} Bot ativo, grupo fecha e abre automaticamente.}

\vspace{10pt}

% ── Fase 2 ───────────────────────────────────────────────
\subsection{Fase 2 · Engajamento — Leaderboard \hfill {\normalfont\small\color{soft}22–23/02}}

{\color{mid}
Adicionar rastreamento de mensagens e exibição do ranking.
Entrega combinada com a Fase 1 como requisito mínimo acordado.
}

\begin{itemize}[leftmargin=1.5em, itemsep=3pt]
  \item {\color{ink}RF2} — Contagem de mensagens por membro no período
  \item {\color{ink}RF2} — Comando \texttt{/top} para exibição manual do Top 10
  \item {\color{ink}RF2} — Envio automático do ranking na reabertura do grupo
\end{itemize}

\vspace{4pt}
{\small\color{green}$\checkmark$ \textbf{Marco:} Requisito mínimo (RF1 + RF2) entregue até 23/02.}

\vspace{10pt}

% ── Fase 3 ───────────────────────────────────────────────
\subsection{Fase 3 · Retenção — Inatividade e Avisos \hfill {\normalfont\small\color{soft}24–26/02}}

{\color{mid}
Implementar políticas automáticas de engajamento e punição progressiva.
Depende de persistência de dados implementada na Fase 2.
}

\begin{itemize}[leftmargin=1.5em, itemsep=3pt]
  \item {\color{ink}RF3.1} — Aviso privado após 3 dias de inatividade
  \item {\color{ink}RF3.2} — Banimento automático após 7 dias de inatividade
  \item {\color{ink}RF4} — Sistema de warnings acumulativos com punições automáticas
\end{itemize}

\vspace{4pt}
{\small\color{yellow}$\circ$ \textbf{Marco:} Moderação proativa funcionando sem intervenção manual.}

\vspace{10pt}

% ── Fase 4 ───────────────────────────────────────────────
\subsection{Fase 4 · Proteção — Filtro de Conteúdo \hfill {\normalfont\small\color{soft}27–28/02}}

{\color{mid}
Camada de segurança automática. Fase de maior complexidade — escopo pode ser
reduzido se o tempo for insuficiente, priorizando RF5.2 e RF5.3.
}

\begin{itemize}[leftmargin=1.5em, itemsep=3pt]
  \item {\color{ink}RF5.2} — Remoção de links suspeitos \textit{(prioridade alta)}
  \item {\color{ink}RF5.3} — Bloqueio de divulgação de outros grupos \textit{(prioridade alta)}
  \item {\color{ink}RF5.1} — Filtro de palavrões \textit{(prioridade média)}
  \item {\color{ink}RF5.4} — Detecção de ataques pessoais \textit{(prioridade baixa / pós-entrega)}
\end{itemize}

\vspace{4pt}
{\small\color{yellow}$\circ$ \textbf{Marco:} Grupo protegido contra spam e conteúdo indesejado.}

\vspace{16pt}

% ── Tabela resumo ────────────────────────────────────────
{\color{rule}\hrule}
\vspace{8pt}
{\small
\begin{tabularx}{\textwidth}{@{} l l l X @{}}
  \color{soft}\textbf{Fase} & \color{soft}\textbf{Período} & \color{soft}\textbf{RFs} & \color{soft}\textbf{Status} \\[4pt]
  \color{ink}1 · MVP         & \color{mid}20–21/02 & \color{purple}RF1          & \color{green}Comprometido \\
  \color{ink}2 · Leaderboard & \color{mid}22–23/02 & \color{purple}RF2          & \color{green}Comprometido \\
  \color{ink}3 · Retenção    & \color{mid}24–26/02 & \color{purple}RF3 + RF4    & \color{yellow}Melhor esforço \\
  \color{ink}4 · Proteção    & \color{mid}27–28/02 & \color{purple}RF5          & \color{yellow}Melhor esforço \\
\end{tabularx}
}

\vspace{16pt}
{\footnotesize\color{soft} GuroZord · Documento interno · 2026 · Feito por \textcolor{purple}{GuroNaive}}

\end{document}